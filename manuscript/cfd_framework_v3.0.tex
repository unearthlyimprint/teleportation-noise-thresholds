\documentclass[twocolumn,10pt]{article}
\usepackage[utf8]{inputenc}
\usepackage{amsmath,amssymb,amsthm}
\usepackage{graphicx}
\usepackage{hyperref}
\usepackage{physics}
\usepackage{geometry}
\usepackage{booktabs}
\usepackage{xcolor}
\usepackage{caption}
\geometry{margin=0.75in}

\newtheorem{theorem}{Theorem}
\newtheorem{conjecture}[theorem]{Conjecture}
\newtheorem{proposition}[theorem]{Proposition}
\newtheorem{definition}[theorem]{Definition}
\newtheorem{remark}[theorem]{Remark}

\title{\textbf{Coherence Field Dynamics:\\
An Information-Geometric Framework for Quantum Decoherence\\with Hardware Validation}\\[0.3em]
\normalsize v1.0 --- Theoretical Framework with Cross-Platform Experimental Evidence}

\author{Celal Arda\\
\small Independent Researcher, Computational Foundations of Quantum Gravity\\
\small \texttt{celal.arda@outlook.de}}

\date{February 23, 2026}

\begin{document}
\maketitle

\begin{abstract}
Standard holographic frameworks (AdS/CFT, Ryu-Takayanagi) encode bulk geometry through entanglement entropy alone. We introduce Coherence Field Dynamics (CFD), a theoretical framework in which quantum coherence---the off-diagonal density matrix elements---serves as an independent geometric degree of freedom encoding effective bulk depth, complementary to entanglement's encoding of connectivity. The coherence field $\phi(p,\gamma)$ satisfies a wave equation in information-geometric parameter space, generating a Lorentzian metric $ds^2 = \phi^2(dp^2 - d\gamma^2)$ that degenerates at a critical decoherence threshold $\gamma_c$, creating an information horizon. We derive $\gamma_c$ from geometric stability analysis and establish a correspondence between MERA tensor network depth and effective decoherence in reduced descriptions. We present five testable predictions. Companion experimental work demonstrates a fidelity threshold at $\gamma \approx 0.535$ on the IonQ platform ($F = 0.988 \pm 0.003$ baseline, collapse to $F \approx 0$ under dephasing) and consistency with the dephasing response on Pasqal's FRESNEL\_CAN1 neutral-atom QPU (noise ratio $1.13\times$ at $\gamma = 0.20$). These results are consistent with CFD's predicted critical behavior, though they do not uniquely distinguish CFD from standard noise models. Four predictions---coherence-depth complementarity, MERA-decoherence correspondence, coherence-modulated entropy scaling, and hardware validation of the reversibility property---remain open and experimentally accessible.
\end{abstract}

% ============================================================
\section{Introduction}
% ============================================================

\subsection{Motivation}

The holographic principle, formalized through the AdS/CFT correspondence~\cite{Maldacena1999}, posits a duality between quantum field theories on a boundary and gravitational theories in a higher-dimensional bulk. The Ryu-Takayanagi (RT) formula~\cite{Ryu2006}
\begin{equation}
    S_{\text{boundary}} = \frac{A_{\min}}{4G_N}
    \label{eq:RT}
\end{equation}
relates entanglement entropy to minimal bulk surface area. Multiscale Entanglement Renormalization Ansatz (MERA) tensor networks~\cite{Vidal2007, Swingle2012} provide a discrete realization, mapping renormalization group flow to holographic depth~\cite{Beny2013, Evenbly2015}.

However, standard holography treats entanglement entropy as the \emph{primary} quantum information observable. Quantum coherence---the off-diagonal density matrix elements---remains geometrically uninterpreted, despite being an independent resource under LOCC operations~\cite{Streltsov2017, Baumgratz2014}. This omission is significant: two quantum states can share identical entanglement entropy yet differ dramatically in their coherence properties, suggesting that entanglement entropy alone cannot fully encode the geometry of information space.

\subsection{Key Thesis: Coherence-Depth Complementarity}

\begin{conjecture}[Coherence-Depth Complementarity]
\label{conj:complementarity}
Quantum coherence $\phi(p, \gamma)$ parameterizes information geometry such that:
\begin{itemize}
    \item Entanglement entropy $S(\rho)$ determines bulk \textbf{connectivity} (minimal surface area).
    \item Coherence field $\phi$ determines effective bulk \textbf{depth} (information accessibility).
\end{itemize}
\end{conjecture}

This extension introduces three new elements beyond standard holography:
\begin{enumerate}
    \item A coherence-modulated entropy relation with attenuation function $f(\gamma)$.
    \item An effective MERA-decoherence correspondence in reduced descriptions.
    \item Critical behavior at a decoherence threshold $\gamma_c$, whose specific value depends on the system geometry.
\end{enumerate}

All three are testable on current quantum hardware.

% ============================================================
\section{Theoretical Framework}
% ============================================================

\subsection{Coherence Field Equation}

The coherence field $\phi : (p, \gamma) \to \mathbb{R}$ satisfies a wave-like equation in information geometry:
\begin{equation}
    \Box\phi + m_{\text{eff}}^2 \phi = 0, \qquad \Box = \frac{\partial^2}{\partial p^2} - \frac{\partial^2}{\partial \gamma^2}
    \label{eq:field}
\end{equation}
where $p \in [0,1]$ is the entanglement parameter, $\gamma \geq 0$ is decoherence strength, and $m_{\text{eff}}^2 < 0$ for the coherent regime (solutions decay in $\gamma$).

\begin{remark}
The negative effective mass-squared corresponds to exponential decay in decoherence space, consistent with quantum information-theoretic constraints~\cite{Zurek2003}. This differs from tachyonic instabilities in relativistic field theory; here, the ``decay'' reflects the physical fact that coherence is fragile.
\end{remark}

\subsection{Information Geometry Metric}

The information geometry is equipped with metric~\cite{Amari2016, Braunstein1994}:
\begin{equation}
    ds^2 = g_{\mu\nu}\, dx^\mu dx^\nu = \phi^2(p,\gamma)\left(dp^2 - d\gamma^2\right)
    \label{eq:metric}
\end{equation}

\begin{definition}[Geometric Stability]
The metric signature is characterized by:
\begin{equation}
    \sigma(\gamma) = \det(g_{\mu\nu}) = -\phi^4(p,\gamma)
    \label{eq:sigma}
\end{equation}
Lorentzian signature requires $\sigma < 0$. The stability threshold occurs when $|\sigma| \to 0$ (metric degeneracy).
\end{definition}

\textit{Physical interpretation:} As $\gamma$ increases, $\phi \to 0$ causes metric degeneracy---information becomes geometrically inaccessible. This parallels causal disconnection in relativistic horizons~\cite{VanRaamsdonk2010}.

\subsection{Variational Principle}

We derive effective spacetime geometry from a coherence action principle. The total action couples Einstein gravity to an information-geometric Lagrangian:
\begin{equation}
    S = \int d^4x \sqrt{-g}\left[\frac{c^4}{16\pi G}\mathcal{R} + \mathcal{L}_{\text{coherence}}\right]
\end{equation}
where the coherence Lagrangian is:
\begin{equation}
    \mathcal{L}_{\text{coherence}} = \tfrac{1}{2} F^{\mu\nu}(\partial_\mu\phi)(\partial_\nu\phi) - V(\phi, \mathrm{Tr}(F))
\end{equation}
Here $F^{\mu\nu}$ is the coherence tensor (inverse of Fisher metric $F_{\mu\nu}$), and $\phi(p,\gamma)$ encodes quantum state amplitudes. Variation with respect to $g_{\mu\nu}$ yields modified Einstein equations:
\begin{equation}
    G_{\mu\nu} = \frac{8\pi G}{c^4}\, T^{(\text{eff})}_{\mu\nu}
\end{equation}
with effective stress-energy:
\begin{equation}
    T^{(\text{coherence})}_{\mu\nu} = (\partial_\mu\phi)(\partial_\nu\phi) - g_{\mu\nu}\!\left[\tfrac{1}{2}g^{\alpha\beta}(\partial_\alpha\phi)(\partial_\beta\phi) - V\right]
\end{equation}

\textbf{Analogue gravity interpretation:} This variational principle is an \emph{analogue gravity model}, analogous to acoustic metrics in fluid dynamics~\cite{VanRaamsdonk2010}. The constants $c$ and $G$ appearing here are \emph{effective, system-specific parameters} that map the information-geometric structure, not claims that quantum decoherence bends physical 4D spacetime. The 4D integral provides a template for how coherence-sourced stress-energy \emph{would} modify geometry if promoted to a physical field theory; the present work treats this coupling as a formal analogy.

\subsection{Fisher Information Metric}

The quantum Fisher information matrix on parameter space $\theta = (p, \gamma)$ is:
\begin{equation}
    g^{\text{Fisher}}_{\mu\nu}(\theta) = \mathrm{Tr}\!\left[\rho_\theta\, \partial_\mu\ln\rho_\theta\, \partial_\nu\ln\rho_\theta\right]
\end{equation}
Following Caticha's entropic gravity~\cite{Caticha2005}, the Fisher metric is the emergent spacetime metric. As $\gamma \to \gamma_c$, the Ricci scalar diverges, signaling horizon formation.

% ============================================================
% NOTE: The F_bond quantum fidelity-bonding correspondence,
% which connects information geometry to molecular orbital theory,
% is developed in a separate dedicated paper. See companion work
% on molecular systems (13+ systems analyzed).
% ============================================================

% ============================================================
\section{Holographic Extensions}
\label{sec:holographic}
% ============================================================

\subsection{Coherence-Modulated Entropy Relation}

\begin{proposition}[Coherence Attenuation]
Effective entropy accessible from boundary measurements exhibits coherence-dependent suppression:
\begin{equation}
    S_{\text{eff}}(\gamma) = S_0 \cdot f(\gamma)
    \label{eq:seff}
\end{equation}
where the attenuation function is:
\begin{equation}
    f(\gamma) = \exp(-\gamma / \gamma_c)
    \label{eq:attenuation}
\end{equation}
with critical scale $\gamma_c$ determined by geometric stability analysis.
\end{proposition}

\textit{Physical interpretation:} Unlike standard RT which computes static minimal surfaces, $S_{\text{eff}}(\gamma)$ represents entropy extractable via coherence-preserving measurements. Decoherence effectively ``hides'' bulk entropy from boundary observers~\cite{Breuer2002}.

\textit{Relation to bulk geometry:} In the AdS/CFT context, this suggests:
\begin{equation}
    A_{\text{eff}}(\gamma) = A_{\min} \cdot f(\gamma)
    \label{eq:aeff}
\end{equation}
The effective entangling surface area contracts---not due to topology change, but due to reduced information accessibility through coherent channels.

The critical value $\gamma_c$ emerges from geometric stability requirements. Setting $|\sigma| = \epsilon_{\min}$ (minimal detectable signature) in Eq.~\eqref{eq:sigma}:
\begin{equation}
    \phi^4(\gamma_c) = \epsilon_{\min}
\end{equation}
For Bell states with decoherence model $\phi(\gamma) = \exp(-\gamma/\gamma_0)$:
\begin{equation}
    \gamma_c = \gamma_0 \ln\left(\epsilon_{\min}^{-1/4}\right)
    \label{eq:gamma_c}
\end{equation}
The framework predicts the \emph{existence} of a geometric threshold $\gamma_c$ at which metric degeneracy occurs and information become inaccessible. The specific numerical value of $\gamma_c$ depends on the system parameters: $\gamma_0$ (the decoherence scale of the physical system) and $\epsilon_{\min}$ (the noise floor).

\textit{Relationship to experimental data:} In the companion teleportation experiment~\cite{Arda2026companion}, fidelity collapses at $\gamma \approx 0.535$. This value is determined by the specific circuit geometry---the $R_z$ rotation angles and distribution coefficients $\xi_j$ that orthogonalize the quantum state at that particular $\gamma$. The CFD framework provides a geometric interpretation of \emph{why} such thresholds occur (metric degeneracy), but the specific threshold value is circuit-dependent, not a universal constant.

\subsection{MERA-CFD Effective Correspondence}

\begin{conjecture}[Effective Decoherence in Reduced Descriptions]
\label{conj:mera}
MERA tensor network depth $k$ maps to effective decoherence in reduced density matrices:
\begin{equation}
    \gamma_{\text{eff}}(k) = \gamma_0 + k \cdot \Delta\gamma, \qquad k = 0, 1, 2, \ldots
    \label{eq:mera}
\end{equation}
where $\Delta\gamma$ quantifies information loss per coarse-graining layer.
\end{conjecture}

\begin{remark}
This correspondence is \emph{effective}, not fundamental~\cite{Orus2014}. MERA uses unitary transformations, while decoherence is non-unitary. The mapping arises because tracing out fine-scale degrees of freedom in MERA mimics decoherence from a reduced subsystem perspective.
\end{remark}

\textit{Operational meaning:} An observer with access only to coarse-grained layers experiences entropy reduction equivalent to decoherence-induced information loss, despite the underlying unitary dynamics.

\textit{Consequence:} MERA's holographic radial direction provides a geometric representation of cumulative effective decoherence---the $\gamma$ axis corresponds to holographic depth in reduced descriptions.

\textit{Status:} This prediction remains experimentally untested. A proposed protocol (Section~\ref{sec:future_mera}) can validate it on current NISQ hardware.

% ============================================================
\section{Testable Predictions}
\label{sec:predictions}
% ============================================================

CFD produces five distinct testable predictions, summarized in Table~\ref{tab:predictions}.

\begin{table*}[t]
    \centering
    \small
    \begin{tabular}{clcl}
        \toprule
        \textbf{No.} & \textbf{Prediction} & \textbf{Status} & \textbf{Platform} \\
        \midrule
        1 & Fidelity threshold at $\gamma_c \approx 0.535$ & \textbf{Consistent} & IonQ simulator + Forte-1 QPU + Pasqal QPU \\
        2 & Unitary reversibility under conjugate dephasing & \textbf{Verified (sim)} & IonQ simulator (trivially, $U^\dagger U = I$) \\
        3 & Coherence-depth complementarity (Conjecture~\ref{conj:complementarity}) & Open & Testable on 3--5 qubit hardware \\
        4 & MERA-decoherence correspondence (Conjecture~\ref{conj:mera}) & Open & Testable on NISQ hardware \\
        5 & Exponential entropy attenuation $S_{\text{eff}} = S_0 e^{-\gamma/\gamma_c}$ & Open & Requires partial tomography \\
        \bottomrule
    \end{tabular}
    \caption{Summary of CFD predictions. Prediction~1 is consistent with experimental data but not uniquely distinguished from standard noise models. Prediction~2 follows from the unitarity of $R_z$ rotations. Three predictions remain open and accessible with current technology.}
    \label{tab:predictions}
\end{table*}

\subsection{Parametric Geometry Evolution}

Unlike static AdS/CFT geometry, CFD predicts continuous parametric evolution of the information-geometric metric:
\begin{equation}
    g_{\mu\nu}(\gamma) = \phi^2(\gamma) \cdot \eta_{\mu\nu}
\end{equation}
where $\eta_{\mu\nu} = \mathrm{diag}(1,-1)$. As $\gamma$ increases from 0 to $\gamma_c$:
\begin{itemize}
    \item Metric determinant: $|\sigma|$ decreases continuously from $\phi_0^4$ to $\epsilon_{\min}$.
    \item Ricci scalar: $\mathcal{R}(\gamma)$ increases, diverging as $\gamma \to \gamma_c$.
    \item Geodesic completeness: Lost at $\gamma_c$ (information horizon).
\end{itemize}

This parametric evolution distinguishes CFD from both standard AdS/CFT (static geometry from fixed boundary state) and standard decoherence models (smooth exponential decay without geometric interpretation).

\subsection{Critical Behavior at Decoherence Threshold}

Near $\gamma_c$, the conformal factor scales as:
\begin{equation}
    \phi(\gamma) = \phi_0 e^{-\alpha\gamma}
\end{equation}
A power-law fit to the fidelity data near the threshold yields an effective exponent $\beta_{\text{eff}} = 1.05 \pm 0.59$. However, this fit is from a 3-qubit system and should be interpreted cautiously---true critical exponents require the thermodynamic limit ($N \to \infty$).

\subsection{Coherence-Dependent Information Accessibility}

CFD predicts a sharp distinction between entanglement persistence and information accessibility:

\begin{conjecture}[Coherence-Information Decoupling]
Near $\gamma_c$, quantum correlations (entanglement) can persist while coherence-mediated information transfer ceases. Specifically, $S(\rho) > 0$ while information accessibility $\to 0$ at $\gamma = \gamma_c$.
\end{conjecture}

This prediction is supported by numerical simulation at $\gamma = 0.533 \approx \gamma_c$: the metric determinant $|\sigma| = 0.0044$ (near-degenerate), yet quantum fidelity $F = 0.734$ remains super-classical ($F > 0.5$). Coherence and entanglement exhibit distinct scaling behaviors near the critical threshold.

% ============================================================
\section{Experimental Validation}
\label{sec:validation}
% ============================================================

Experimental data from companion work~\cite{Arda2026companion} is consistent with several CFD predictions, though the results do not uniquely distinguish CFD from standard noise models. We summarize the key results here.

\subsection{Dephasing Threshold Observation}

\subsubsection{IonQ Trapped-Ion Platform}

Using Azure Quantum's IonQ simulator and Forte-1 QPU, we implemented a 3-qubit teleportation protocol with tunable parametric dephasing ($R_z$ rotations of strength $\gamma$):

\begin{itemize}
    \item \textbf{Simulation:} Fidelity degrades from $F = 0.92$ at $\gamma = 0$ to $F \approx 0$ at $\gamma \approx 0.535$, consistent with the predicted threshold.
    \item \textbf{Hardware baseline:} 3-qubit teleportation on Forte-1 QPU yielded $F = 0.988 \pm 0.003$ (1000 shots), exceeding the classical bound $F_{\text{classical}} = 2/3$ by $48\%$.
    \item \textbf{Control experiment:} Identical circuit without entanglement bridge shows zero information transfer, confirming entanglement as the transfer mechanism.
    \item \textbf{Dephasing-driven collapse:} 9-qubit protocol at $\gamma = 0.535$ yielded $F_{\text{exp}} \approx 0$, consistent with the injected dephasing overwhelming the quantum channel.
\end{itemize}

\subsubsection{Pasqal Neutral-Atom Platform}

The dephasing response was independently tested on Pasqal's neutral-atom architecture:

\begin{itemize}
    \item \textbf{Emulator sweep:} Fine-grained parameter sweep ($\gamma \in [0.15, 0.60]$) on \texttt{EMU\_FREE} reveals a steeper threshold ($\gamma \in [0.15, 0.30]$), likely reflecting differences in noise coupling between the analog Rydberg Hamiltonian and gate-based $R_z$ rotations.
    \item \textbf{Three-tier validation:} The dephasing response is preserved across idealized simulation (\texttt{EMU\_FREE}), exact solver (\texttt{EMU\_SV}), and hardware-realistic emulation (\texttt{EMU\_FRESNEL}).
    \item \textbf{Hardware QPU:} Execution on FRESNEL\_CAN1 QPU (22 atoms, 1500 shots) shows consistency: at $\gamma = 0.20$, QPU ground-state probability ($P_0 = 70.6\%$) matches noiseless simulation ($72.0\%$, noise ratio $1.13\times$).
\end{itemize}

\begin{table}[t]
    \centering
    \small
    \begin{tabular}{lcc}
        \toprule
        \textbf{Experiment} & \textbf{$F$ or $P_0$} & \textbf{Status} \\
        \midrule
        IonQ 3-qubit teleportation & $F = 0.988$ & High fidelity \\
        IonQ control (no ent.) & No transfer & Confirmed \\
        IonQ 9-qubit, $\gamma = 0.535$ & $F \approx 0$ & Collapsed \\
        Pasqal QPU, $\gamma = 0.20$ & $P_0 = 70.6\%$ & Near-ideal \\
        Conjugate dephasing (sim) & $0.00 \to 0.92$ & Recovered (trivially) \\
        \bottomrule
    \end{tabular}
    \caption{Summary of experimental results across platforms. Conjugate dephasing result is simulation-only ($U^\dagger U = I$); all others include hardware data.}
    \label{tab:hardware}
\end{table}

\subsection{Reversibility Under Conjugate Dephasing (Simulation)}

At $\gamma = 0.8$, applying the conjugate dephasing operator $\mathcal{D}^{-1}(\gamma) = \bigotimes_j R_z(-\gamma\pi\xi_j)$ before the dephasing injection restores fidelity from $F = 0.00 \pm 0.01$ to $F = 0.92 \pm 0.04$. This result follows directly from the unitarity of $R_z$ rotations ($U^\dagger U = I$) and therefore does not provide independent evidence for CFD.

\textit{Note:} This reversibility is a mathematical property of unitary operations, not a physical discovery. True decoherence (coupling to an unmeasured environment) is irreversible and cannot be undone by conjugate rotations. The significance of this observation is limited to confirming that the parametric dephasing model is unitary, as designed.

% ============================================================
\section{Discussion}
% ============================================================

\subsection{CFD vs.\ Standard Holography}

\begin{table}[t]
    \centering
    \small
    \begin{tabular}{lp{1.8cm}p{2.0cm}}
        \toprule
        \textbf{Aspect} & \textbf{Standard} & \textbf{CFD} \\
        \midrule
        Info channel & Entanglement only & Ent.\ + Coherence \\
        Geometry & Static surfaces & Parametric \\
        Observable & $S$ (entropy) & $\phi$, $S_{\text{eff}}$ \\
        Bulk evol. & Fixed & $\gamma$-dependent \\
        Testability & Indirect & Direct (hw) \\
        \bottomrule
    \end{tabular}
    \caption{CFD compared to standard holographic principles.}
    \label{tab:comparison}
\end{table}

\subsection{Implications for Quantum Information}

If the open predictions are validated, CFD would establish:
\begin{itemize}
    \item \textbf{Coherence as geometric parameter:} An independent quantum resource with geometric interpretation, complementary to entanglement~\cite{Streltsov2017}.
    \item \textbf{$\gamma_c$ as information horizon:} A threshold beyond which bulk entropy becomes inaccessible in the information-geometric framework.
\end{itemize}

\textit{Speculative connections to ER=EPR and the Black Hole Information Paradox are conceivable but remain far from established. The current experimental circuit (3-qubit teleportation with $R_z$ dephasing) does not contain the scrambling dynamics required for a holographic dual, and any such connections require substantially more theoretical and experimental development.}

\subsection{Limitations and Open Questions}

\begin{enumerate}
    \item \textbf{Many-body scaling:} Current formalism is limited to bipartite systems; multipartite extension requires tensor network generalization~\cite{Orus2014}.
    \item \textbf{Microscopic derivation:} The field equation (Eq.~\ref{eq:field}) is phenomenological; a path integral formulation is needed.
    \item \textbf{Dynamical time evolution:} The framework is parametric in $\gamma$; coupling to external time coordinates requires Hamiltonian formulation~\cite{Breuer2002}.
    \item \textbf{QEC connection:} Relationship between coherence attenuation and code distance in MERA-based error correction remains unexplored.
    \item \textbf{Active Shielding on hardware:} The unitary reversibility result is simulation-only; hardware demonstration would be a strong validation.
\end{enumerate}

\subsection{Proposed Experiments for Open Predictions}
\label{sec:future_mera}

\textbf{Coherence-depth complementarity (Conjecture~\ref{conj:complementarity}):} Prepare pairs of 3--5 qubit states with equal entanglement entropy but different coherence (vary $\gamma$ while compensating $S(\rho)$ via local rotations). Measure information accessibility via quantum state discrimination. If states with lower coherence show reduced accessibility at fixed entropy, the conjecture is confirmed.

\textbf{MERA-decoherence correspondence (Conjecture~\ref{conj:mera}):} Implement a 2--3 layer MERA circuit. Perform partial tomography at each layer to extract $\gamma_{\text{eff}}(k)$. Plot effective decoherence versus layer depth; linearity confirms the correspondence.

\textbf{Entropy attenuation:} Measure accessible entanglement entropy $S_{\text{eff}}(\gamma)$ via entanglement witnesses at multiple $\gamma$ values. Fit to exponential model (Eq.~\ref{eq:attenuation}). This extends existing phase transition data (which measured only ground-state probability) to the entropy observable predicted by CFD.

\subsection{Distinction from Related Work}

CFD differs from Arturo Cerezo Garc\'ia's Radial Coherential Dynamics (RCD)~\cite{Garcia2024}:
\begin{itemize}
    \item \textbf{CFD:} Information geometry in abstract $(p, \gamma)$ parameter space.
    \item \textbf{RCD:} Modified Einstein field equations in physical spacetime coordinates.
\end{itemize}
CFD operates at the quantum information level; RCD modifies gravitational dynamics directly.

% ============================================================
\section{Conclusion}
% ============================================================

Coherence Field Dynamics extends holographic quantum gravity by promoting coherence from a derived quantity to an independent geometric degree of freedom. The framework predicts that information accessibility decreases under decoherence---even when entanglement persists---with a threshold near $\gamma_c \approx 0.535$.

Key contributions of this work:
\begin{enumerate}
    \item \textbf{Theoretical framework:} Coherence field equation, information-geometric metric, and variational principle.
    \item \textbf{Holographic extensions:} Coherence-modulated entropy relation $S_{\text{eff}}(\gamma) = S_0 e^{-\gamma/\gamma_c}$ and MERA-decoherence correspondence $\gamma_{\text{eff}}(k) = \gamma_0 + k \Delta\gamma$.
    \item \textbf{Experimental consistency:} Companion experimental work shows a fidelity threshold at $\gamma \approx 0.535$ on IonQ and a consistent dephasing response on Pasqal FRESNEL\_CAN1 QPU, though these results do not uniquely distinguish CFD from standard noise models. The specific threshold value is determined by circuit geometry, not by CFD alone.
    \item \textbf{Four open predictions:} Coherence-depth complementarity, MERA-decoherence mapping, entropy attenuation, and hardware demonstration of non-trivial reversibility---all experimentally accessible on current NISQ hardware.
\end{enumerate}

The framework's distinguishing feature is testability: unlike approaches that require Planck-scale energies or astronomical observations, CFD predictions can be probed with quantum simulators. The consistency of initial experimental data and the accessibility of the remaining open predictions position CFD as a research program at the intersection of quantum information, holography, and quantum chemistry.

\section*{Acknowledgments}

We thank Microsoft Azure Quantum for computing resources, the IonQ team for simulator and Forte-1 QPU access, and Pasqal for FRESNEL\_CAN1 emulator and QPU access via the Pasqal Cloud platform.

\begin{thebibliography}{99}

\bibitem{Amari2016}
S.~Amari, \textit{Information Geometry and Its Applications}, Applied Mathematical Sciences \textbf{194}, Springer (2016).

\bibitem{Arda2026companion}
C.~Arda, ``Cross-platform noise thresholds in quantum teleportation: Trapped-ion and neutral-atom architectures,'' Preprint v5.0 (2026).

\bibitem{Baumgratz2014}
T.~Baumgratz, M.~Cramer, and M.~B.~Plenio, ``Quantifying coherence,'' Phys.\ Rev.\ Lett.\ \textbf{113}, 140401 (2014).

\bibitem{Beny2013}
C.~B\'eny, ``Causal structure of the entanglement renormalization ansatz,'' New J.\ Phys.\ \textbf{15}, 023020 (2013).

\bibitem{Braunstein1994}
S.~L.~Braunstein and C.~M.~Caves, ``Statistical distance and the geometry of quantum states,'' Phys.\ Rev.\ Lett.\ \textbf{72}, 3439 (1994).

\bibitem{Breuer2002}
H.-P.~Breuer and F.~Petruccione, \textit{The Theory of Open Quantum Systems}, Oxford University Press (2002).

\bibitem{Caticha2005}
A.~Caticha, ``Entropic dynamics,'' AIP Conference Proceedings \textbf{803}, 302 (2005).

\bibitem{Evenbly2015}
G.~Evenbly and G.~Vidal, ``Tensor network renormalization,'' Phys.\ Rev.\ Lett.\ \textbf{115}, 180405 (2015).

\bibitem{Garcia2024}
A.~Cerezo Garc\'ia, ``Radial Coherential Dynamics: A novel approach to quantum gravity,'' Independent research (2024).

\bibitem{Huckel1931}
E.~H\"uckel, ``Quantentheoretische Beitr\"age zum Benzolproblem,'' Z.\ Phys.\ \textbf{70}, 204 (1931).

\bibitem{Jozsa1994}
R.~Jozsa, ``Fidelity for mixed quantum states,'' J.\ Mod.\ Opt.\ \textbf{41}, 2315 (1994).

\bibitem{Maldacena1999}
J.~Maldacena, ``The large $N$ limit of superconformal field theories and supergravity,'' Adv.\ Theor.\ Math.\ Phys.\ \textbf{2}, 231 (1999).

\bibitem{Maldacena2013}
J.~Maldacena and L.~Susskind, ``Cool horizons for entangled black holes,'' Fortschr.\ Phys.\ \textbf{61}, 781 (2013).

\bibitem{Orus2014}
R.~Or\'us, ``A practical introduction to tensor networks,'' Ann.\ Phys.\ \textbf{349}, 117 (2014).

\bibitem{Pauling1939}
L.~Pauling, \textit{The Nature of the Chemical Bond}, Cornell University Press (1939).

\bibitem{Ryu2006}
S.~Ryu and T.~Takayanagi, ``Holographic derivation of entanglement entropy from AdS/CFT,'' Phys.\ Rev.\ Lett.\ \textbf{96}, 181602 (2006).

\bibitem{Streltsov2017}
A.~Streltsov, G.~Adesso, and M.~B.~Plenio, ``Colloquium: Quantum coherence as a resource,'' Rev.\ Mod.\ Phys.\ \textbf{89}, 041003 (2017).

\bibitem{Swingle2012}
B.~Swingle, ``Entanglement renormalization and holography,'' Phys.\ Rev.\ D \textbf{86}, 065007 (2012).

\bibitem{VanRaamsdonk2010}
M.~Van~Raamsdonk, ``Building up spacetime with quantum entanglement,'' Gen.\ Rel.\ Grav.\ \textbf{42}, 2323 (2010).

\bibitem{Vidal2007}
G.~Vidal, ``Entanglement renormalization,'' Phys.\ Rev.\ Lett.\ \textbf{99}, 220405 (2007).

\bibitem{Zurek2003}
W.~H.~Zurek, ``Decoherence, einselection, and the quantum origins of the classical,'' Rev.\ Mod.\ Phys.\ \textbf{75}, 715 (2003).

\end{thebibliography}

\end{document}
